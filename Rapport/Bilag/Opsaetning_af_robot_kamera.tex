\chapter{Opsætning af robotkamera}
\label{appendix:Opsaetning_af_robotkamera}

Dette afsnit vil beskrive hvordan Raspberry Pi'en opsættes til at fungere som en live stream server med Pi-kameraet.

1 - Start med at installere motioneye:
\mybox{wget https://github.com/ccrisan/motioneye/wiki/precompiled/ffmpeg\_3.1.1-1\_armhf.deb\\
	sudo dpkg -i ffmpeg\_3.1.1-1\_armhf.deb}

2 - Derefter skal følgende pakker installeres, som softwaren er afhængig af for at kunne fungere:
\mybox{sudo apt-get install curl libssl-dev libcurl4-openssl-dev libjpeg-dev libx264-142\\ libavcodec56 libavformat56 libmysqlclient18 libswscale3 libpq5}

3 - Nu kan motion installeres:
\mybox{wget https://github.com/Motion-Project/motion/releases/download/release-4.0.1/pi\_jessie\_motion\_4.0.1-1\_armhf.deb\\
sudo dpkg -i pi\_jessie\_motion\_4.0.1-1\_armhf.deb}

4 - Konfigurationen af programmet kan nu laves i filen:
\mybox{sudo nano /etc/motion/motion.conf}
ved at ændre følgende linjer:
\begin{itemize}
	\item deamon on
	\item stream\_localhost off
	\item output\_pictures off
	\item ffmpeg\_output\_movies off
	\item stream\_maxrate 100 (vil gøre streaming live, men vil bruge båndbredde og ressourcer)
	\item framerate 100
	\item width 640
	\item height 480
\end{itemize}


5 - Programmet motion kan nu konfigureres ved at redigere denne fil:
\mybox{sudo nano /etc/default/motion\\
ved at ændre linjen fra no til yes\\
start\_motion\_daemon=yes}

6 - Sørg for at programmet starter automatisk ved boot. Dette gøres ved at tilføje pi-cam i:
\mybox{sudo raspi-config}

7 - Efter at have genstartet Raspberry Pi’en vil der nu blive streamet på port 8081.
