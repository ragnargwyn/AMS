%%---------------------------------------------------------------------
%	Preamble
%	Semesterprojekt 4 - gruppe 1
%	IHA F17
%---------------------------------------------------------------------
\documentclass[12pt,fleqn,a4paper]{report}
\usepackage[utf8]{inputenc}
\usepackage[danish]{babel}
\usepackage[top=2.5cm, left=2cm, right=2cm, bottom=2.5cm]{geometry}
\usepackage{graphicx}
\usepackage[bottom]{footmisc}
\usepackage{framed}
\usepackage{caption}
\usepackage{float}
\usepackage{mdframed}
\usepackage{listings}
\usepackage{color}
\usepackage[T1]{fontenc}
\usepackage{amsmath,amsfonts,amsthm} % Math packages
\usepackage{array}
\usepackage{wrapfig}
\usepackage{multirow}
\usepackage{tabu}
\usepackage{longtable}
\usepackage{lastpage}
\usepackage{fancyhdr}
\usepackage[compact]{titlesec}
\usepackage[table,xcdraw]{xcolor}
\usepackage{arydshln}
\usepackage[style=ieee]{biblatex}

\definecolor{mygreen}{RGB}{28,172,0} % color values Red, Green, Blue
\definecolor{mylilas}{RGB}{170,55,241}
\renewcommand{\lstlistingname}{Kodeudsnit}
\tabulinesep=3mm

\setcounter{secnumdepth}{2}
\setcounter{tocdepth}{1}

\setlength{\parindent}{0mm} %intet indryk
\setlength{\parskip}{3mm} 	%linjeskift v. afsnit

% Ændring af enumerize og itemize 
\usepackage{enumitem} % @http://ctan.org/pkg/enumitem
\setlist[itemize]{topsep=0pt, itemsep=0.5pt}
\setlist[enumerate]{topsep=0pt, itemsep=0.5pt}

%afstand omkring sections
\titlespacing{\section}{0pt}{5mm}{0pt}
\titlespacing{\subsection}{0pt}{2mm}{0pt}
\titlespacing{\subsubsection}{0pt}{2mm}{0pt}

\usepackage{arydshln}
%aryd
\setlength\dashlinedash{3pt}
\setlength\dashlinegap{4pt}

\lstset{language=C++,
	breaklines=true,
	keywordstyle=\color{blue},
	stringstyle=\color{red},
	commentstyle=\color{mygreen},
	morecomment=[l][\color{magenta}]{\#}
}

%header & footer
\makeatletter
\pagestyle{fancy}
\fancypagestyle{plain}{}
\renewcommand{\chaptermark}[1]{\markboth{#1}{}}
\setlength{\headheight}{35pt}
\fancyfoot{} % clear all fields
\fancyfoot[R]{Side \thepage\ af \pageref{LastPage}}
\fancyhead{} % clear all fields
\fancyhead[L]{\includegraphics[clip, trim = 0 0 240pt 0, height=30pt]{Figur/IHA_AU_logo.png}}
\fancyhead[R]{\includegraphics[height = 30pt]{Figur/logo.png}}
\fancyhead[C]{\ifnum\value{chapter}>0 \leftmark \fi}
\renewcommand{\headrulewidth}{0pt}

\def\thickhrulefill{\leavevmode \leaders \hrule height 1.2ex \hfill \kern \z@}
\def\@makechapterhead#1{
  \vspace*{10\p@}%
  {\parindent \z@ \centering \reset@font
        \thickhrulefill\quad 
        \scshape\bfseries\textit{\@chapapp{}  \thechapter}  
        \quad \thickhrulefill
        \par\nobreak
        \vspace*{10\p@}%
        \interlinepenalty\@M
        \hrule
        \vspace*{10\p@}%
        \Huge \bfseries #1 \par\nobreak
        \par
        \vspace*{10\p@}%
        \hrule
        \vskip 40\p@
  }}



\graphicspath{ {Figur/} }


%Se Kodeudsnit \ref{lstlisting:generel_kode}

%\captionof{lstlisting}{Generelle egenskaber for koden til fremstilling af diverse figure i matlab} 
%\label{lstlisting:generel_kode}
%\vspace{5mm} %5mm vertical space
%
%\subsection{Kode til lyd i forhold til tiden}
%\begin{framed}
%\begin{center}
%\begin{lstlisting}
%figure('name','trafikstoejen i fuld laengde'); clf
%subplot(211);
%plot(t,s_sound_left)
%xlabel('Tid (sek)')
%ylabel('Signalstyrke')
%title('Trafikstoej set i forhold til tiden')
%grid on
%hold on
%\end{lstlisting}
%\end{center}
%\end{framed}




%\begin{document}

\chapter{Opnåede erfaringer}
Efter afslutning af dette projekt, er der reflekteret over gruppens opnåede erfaringer. \\
Fra projektets start var der fokus på, at projektet skulle inddeles i forskellige arbejdsområder, således at gruppen kunne arbejde i mindre hold af en til to personer. 
Dette har fungeret godt og har haft en positiv indflydelse på udviklingshastigheden. \\
Opdelingen stillede dog det krav, at der skulle defineres nogle klare grænseflader - både software- og hardwaremæssigt. 
Under forløbet er det erfaret at dette kunne være gjort bedre, hvilket resulterede i problemer under implementeringen.

Beslutningen om at arbejde i mindre grupper betød også, at indsatsen i  arkitekturfasen skulle forøges, så der ikke kunne opstå usikkerheder ved implementeringen blandt gruppemedlemmerne.
Dette kom til udtryk, da den tiltænkte master-slave rollefordeling for I2C forbindelsen viste sig, ikke at kunne lade sig gøre. 
Dette blev opdaget sent i forløbet, fordi hver gruppe havde fokuseret på at få sit eget ansvarsområde til at fungere. \\
En sådan fejl vil også kunne være fundet, hvis en dybere foranalyse var foretaget. Projektet ville på flere områder have nydt godt af en bedre foranalyse, så misforståelser og fejl kunne forebygges.

Der opstod i projektet også en række af problemer, grundet manglende kommunikation mellem gruppens medlemmer.
Det mest udtalte problem var, at gruppen ikke fik specificeret versioner af de valgte udviklingsværktøjer som skulle bruges. 
Dette gav problemer i integrationsfasen, da de sammensatte moduler ikke virkede som forventet på andre versioner af udviklingsværktøjet.

Et stort ønske efter forløbet er muligheden for at kunne lave en mere iterationsbaseret udvikling. 
Muligheden for at kunne foretage tests løbende, specielt integrationstest imellem gruppernes moduler, kunne have hjulpet med at finde bedre løsninger samt at sortere overflødige dele fra.

Det kan i retrospektiv ses at kode-review på tværs af arbejdsgrupper, ville have været en fordel, da dette kunne have fanget fejl og givet optimeringsmuligheder.

%\end{document}