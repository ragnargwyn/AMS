%---------------------------------------------------------------------
%	Preamble
%	Semesterprojekt 4 - gruppe 1
%	IHA F17
%---------------------------------------------------------------------
\documentclass[12pt,fleqn,a4paper]{report}
\usepackage[utf8]{inputenc}
\usepackage[danish]{babel}
\usepackage[top=2.5cm, left=2cm, right=2cm, bottom=2.5cm]{geometry}
\usepackage{graphicx}
\usepackage[bottom]{footmisc}
\usepackage{framed}
\usepackage{caption}
\usepackage{float}
\usepackage{mdframed}
\usepackage{listings}
\usepackage{color}
\usepackage[T1]{fontenc}
\usepackage{amsmath,amsfonts,amsthm} % Math packages
\usepackage{array}
\usepackage{wrapfig}
\usepackage{multirow}
\usepackage{tabu}
\usepackage{longtable}
\usepackage{lastpage}
\usepackage{fancyhdr}
\usepackage[compact]{titlesec}
\usepackage[table,xcdraw]{xcolor}
\usepackage{arydshln}
\usepackage[style=ieee]{biblatex}

\definecolor{mygreen}{RGB}{28,172,0} % color values Red, Green, Blue
\definecolor{mylilas}{RGB}{170,55,241}
\renewcommand{\lstlistingname}{Kodeudsnit}
\tabulinesep=3mm

\setcounter{secnumdepth}{2}
\setcounter{tocdepth}{1}

\setlength{\parindent}{0mm} %intet indryk
\setlength{\parskip}{3mm} 	%linjeskift v. afsnit

% Ændring af enumerize og itemize 
\usepackage{enumitem} % @http://ctan.org/pkg/enumitem
\setlist[itemize]{topsep=0pt, itemsep=0.5pt}
\setlist[enumerate]{topsep=0pt, itemsep=0.5pt}

%afstand omkring sections
\titlespacing{\section}{0pt}{5mm}{0pt}
\titlespacing{\subsection}{0pt}{2mm}{0pt}
\titlespacing{\subsubsection}{0pt}{2mm}{0pt}

\usepackage{arydshln}
%aryd
\setlength\dashlinedash{3pt}
\setlength\dashlinegap{4pt}

\lstset{language=C++,
	breaklines=true,
	keywordstyle=\color{blue},
	stringstyle=\color{red},
	commentstyle=\color{mygreen},
	morecomment=[l][\color{magenta}]{\#}
}

%header & footer
\makeatletter
\pagestyle{fancy}
\fancypagestyle{plain}{}
\renewcommand{\chaptermark}[1]{\markboth{#1}{}}
\setlength{\headheight}{35pt}
\fancyfoot{} % clear all fields
\fancyfoot[R]{Side \thepage\ af \pageref{LastPage}}
\fancyhead{} % clear all fields
\fancyhead[L]{\includegraphics[clip, trim = 0 0 240pt 0, height=30pt]{Figur/IHA_AU_logo.png}}
\fancyhead[R]{\includegraphics[height = 30pt]{Figur/logo.png}}
\fancyhead[C]{\ifnum\value{chapter}>0 \leftmark \fi}
\renewcommand{\headrulewidth}{0pt}

\def\thickhrulefill{\leavevmode \leaders \hrule height 1.2ex \hfill \kern \z@}
\def\@makechapterhead#1{
  \vspace*{10\p@}%
  {\parindent \z@ \centering \reset@font
        \thickhrulefill\quad 
        \scshape\bfseries\textit{\@chapapp{}  \thechapter}  
        \quad \thickhrulefill
        \par\nobreak
        \vspace*{10\p@}%
        \interlinepenalty\@M
        \hrule
        \vspace*{10\p@}%
        \Huge \bfseries #1 \par\nobreak
        \par
        \vspace*{10\p@}%
        \hrule
        \vskip 40\p@
  }}



\graphicspath{ {Figur/} }


%Se Kodeudsnit \ref{lstlisting:generel_kode}

%\captionof{lstlisting}{Generelle egenskaber for koden til fremstilling af diverse figure i matlab} 
%\label{lstlisting:generel_kode}
%\vspace{5mm} %5mm vertical space
%
%\subsection{Kode til lyd i forhold til tiden}
%\begin{framed}
%\begin{center}
%\begin{lstlisting}
%figure('name','trafikstoejen i fuld laengde'); clf
%subplot(211);
%plot(t,s_sound_left)
%xlabel('Tid (sek)')
%ylabel('Signalstyrke')
%title('Trafikstoej set i forhold til tiden')
%grid on
%hold on
%\end{lstlisting}
%\end{center}
%\end{framed}




\begin{document}

\section{Skærm}
Skærmen er den grafiske brugergrænseflade til systemet. Den skal give brugeren en grafisk præsentation af lydinputtet i mikrofonen. Lyden bliver vist på skærmen ved hjælp af et spektrogram der er 40 pixels bredt, og 240 pixels højt.
Skærmen er inddelt i 8 søjler på hver 40 pixels. Den første søjle viser farvespektret hvor det kan ses hvilken farve der beskriver lyddensiteten. De sidste 7 bins viser spektret over det aktuelle lydbillede.

\begin{figure} [H]
	\centering
	\includegraphics[width=0.7 \textwidth]{skaerm_spectrum.png}
	\captionof{figure}{Brugergrænsefladen med 8 bins. 7 der hver repræsenterer lydbilledet for tiden, og et bin der viser farvespektret}
	\label{fig:skaerm_spectrum}
\end{figure}

\subsection{Farve}
Farverne som det aktuelle spektrogram bin der vises, bliver valgt ud fra outputtet fra FHT modulet. Outputtet fra FHT modulet er et array med 128 pladser der hver indeholder et tal mellem 0 og 255. 0 betyder lav densitet og 255 betyder høj.
For at få dette tal repræsenteret som en farve på displayet, skal tallet omsættes til en RGB565 kode.
Dette er gjort ved at lave en lookup table på 3X256 pladser, som et python script autogenerer og derved laver en c fil.

\begin{figure} [H]
	\centering
	\includegraphics[width=0.7 \textwidth, trim={0cm, 2cm, 0cm, 1.8cm}, clip]{rgb_diagram.pdf}
	\captionof{figure}{RGB farveblandingsdiagram der bliver brugt til algoritmen til lookup table}
	\label{fig:rgb_diagram}
\end{figure}


På figur \ref{fig:rgb_diagram} ses diagrammet over farveblandingen der bliver brugt til at lave lookup tabellen. X-aksen er det tal der bliver slået op med, altså outputtet fra FHT modulet. Y-aksen viser RGB koden.




\end{document}