\section{Diskussion}
Sammenlignet med ideen bag projekten, fungerer systemet som ønsket, bortset fra loggen, som ikke er helt færdig. 
Der er dog nogle beslutninger i designet, som kan diskuteres. 

For at få et fuldt spektrum med ift. den menneskelige hørelse, kunne samplefrekvensen have været højere. 
Fx 48 $kHz$. 
Frekvensopløsningen, ville dog blive ringere. Til gengæld kunne FHT bibliotekets indbyggede oktav-output bruges. Dette ville otte bins med frekvensafhængig størrelse.

Anti-aliaseringen i systemet kunne med fordel gøres stejlere. Det ses tydeligt ved sweep-signaler over $10 kHz$, at indholdet spejles ned nytteområdet. 

\subsection{Fremtidigt arbejde}
De første to features, som skulle laves, ville være:
En log der gemmes på et SD kort.
Fastlagt opdateringsrate på displayet. Dette ville kunne implementeres med en timer. 

Skulle der arbejdes videre med projektet, kunne nogle af systemets elementer med fordel skiftes ud. 
En dedikeret audio codek ville kunne sample med en større samplefrekvens i højere opløsning. 
Dette ville forbedre systemet markant. 
Dog skulle der også bruges en længere algoritme, før den forbedrede opløsning ikke gør frekvensopløsningen dårligere.

Skærmen kunne skiftes ud til fordel for en enhed med større farvedybde, evt. RGB i 888 format i stedet for 565. 
Derudover kunne højere opløsning medvirke, at der ikke gik information tabt, i form af de 8 frekvensbins, der ignoreres. 
En større skærmopløsning kræver dog et bredere interface, så skærmen kan opdateres hurtigere. 

Alt i alt ville disse opgraderinger højst sandsynligt resultere i, at systemet ikke ville kunne implementeres til en mega2560.

\section{Konklusion}
Det er i projektet lykkedes at implementere et realtidsspektrogram til en arduino2560. 
Spektrogrammet udskrives løbende på et farvedisplay med en intuitivt forståelig gradient.  
Log-systemet, som skulle skrive data til et SD kort, virker dog ikke. 

Opløsningen i tidsdomænet, såvel som i frekvensdomænet er lavere end ønsket, men resultatet giver et godt billede af støjniveau og frekvensindhold i lydsignalet. 

Ud fra hukommelses-forbruget, og de øvrige begrænsninger, kan systemets ydeevne ikke hæves betydeligt, uden at skulle skifte processor. 
 
