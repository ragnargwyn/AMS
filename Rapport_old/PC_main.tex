\section{PC main}
Main har til ansvar at binde de forskellige moduler i systemet sammen, så beskeder ender de rigtige steder.
Derudover skal programmets opgaver fordeles ud på et passende antal tråde, mens der sørges for, at trådene ikke ødelægger hinandens opgaver.\\
Opgaverne der udføres på PC'en, for at projektet kan køre, er følgende:
	\begin{itemize}
	\item Vise grafisk interface
	\item Køre eyetracking, så der kan udlæses koordinater
	\item Udlæse øjenretning med eyetracking og flytte cursoren tilsvarende
	\item Oprette Wifi-forbindelse mellem PC og Kontrolenhed
	\item Reagere på interaktion med GUI'et
	\item Læse beskeder fra Kontrolenheden og reagere på disse
	\end{itemize}

Til udviklingen af GUI'et, er det valgt at anvende Qt Creators GUI-værktøj.
For at eksekvere GUI'et er det et krav fra Qt's side, at eksekveringen sker fra main-tråden. 
Dette gør, at al funktionalitet, der skal ske mens GUI'et bliver vist, skal køre i tråde.
I implementeringen er det derfor valgt at give hver af ovenstående ansvarspunkter sin egen tråd. 
Dette gøres, da nogen af opgaverne skal køre konstant - og gerne parallelt - for maksimal ydeevne.
Andre er reaktive opgaver, der kræver en tråd, som venter på et input.

I de følgende afsnit vil interaktionen mellem trådene blive beskrevet.

\subsection{Styring af cursor med øjendetekteringsmodul}
For at flytte cursoren ved brug af øjendetekteringsmodulet oprettes to tråde: \\
Den ene tråd har til ansvar at køre øjendetekteringsmodulet.\\
Den anden tråd har til ansvar, at udlæse brugerens synsretning fra øjendetekteringsmodulet, matche retningen med koordinater på GUI'et og flytte cursoren tilsvarende.

Koden til dette ansvarsområde kan findes i trådfunktionerne \textit{eyeTrackingFunc()} og \textit{moveCursorFunc()}.


\subsection{Oprettelse af Wifi-forbindelse}
Da oprettelsen af Wifi-forbindelsen skal ske samtidig med at GUI'et kører, oprettes en tråd, som opretter forbindelse til robotten, hvorefter den nedlægges. 
Dette sker ved at kalde Wifi-klassens connectToPc()-funktion med robottens IP og en default port med nummer 4567. 
Hvis dette ikke lykkes tælles portnummeret én op, hvorefter forbindelsen forsøges oprettet igen efter et sekund.
Dette fortsætter, indtil der er oprettet forbindelse, hvorefter brugeren informeres om, at forbindelsen er oprettet.

Koden til dette ansvarsområde findes i trådfunktionen \textit{establishConnectionToRobot()}.


\subsection{Reaktion på interaktion med GUI}
For at kunne reagere på interaktion med GUI'et, er programmet designet således, at der signaleres til main, hvis der sker en ændring på et vilkårligt valgfelt. 
Signalerne bliver derefter læst af en eventdrevet tråd, hvis eneste ansvar er at læse signalerne og videresende disse til robotten via Wifi. 

Signalerne er implementeret med en instans af klassen MsgQueue, der er delt mellem main og GUI'et.
Ved at anvende MsgQueue mindskes risikoen for forkert timing, da denne klasse er lavet til at signalere mellem tråde.
For yderligere dokumentation henvises til sektionen om MsgQueue-klassen. \\
For at reagere på events, er der i tråden implementeret et event-loop, der har til formål at læse, håndere og derefter slette en besked fra MsgQueue'en.\\
I GUI-klassen vil der kunne findes kode, der i tilfælde af en valgfeltsændring opretter en besked, hvorefter den indsættes i MsgQueue'en.

Koden til dette ansvarsområde findes i trådfunktionen \textit{clickRespondFunc()}.

\subsection{Reaktion på beskeder fra robot}
For at kunne reagere på beskeder fra robotten, oprettes en tråd, hvis ansvar er at læse fra Wifi-forbindelsen og reagere tilsvarende.
I projektet kan der modtages tre forskellige beskeder fra robotten: 'i bevægelse', 'stoppet' og 'gået i dvale'.
De to første beskeder udløser en besked til GUI'et om, at ændre menu-/bremseikonet.
Den sidste lukker programmet ned på PC-siden.

Koden til dette ansvarsområde findes i trådfunktionen \textit{receiveFromRobotFunc()}.