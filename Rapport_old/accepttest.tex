%\begin{document}

\chapter{Accepttest}

Accepttesten af systemet er udformet efter de funktionelle og ikke funktionelle krav. Gennemgangen af denne test kan ses i bilag \ref{appendix:Accepttest}. 

Prototypen er i stand til at bestå alle de testbare krav som er stillet til den. 
Dog fejles nogen af punkterne, men disse er ikke af vital betydning for prototypen. 

\subsubsection{Diskussion af Accepttest}
Robottens max. hastighed er markant lavere end den angivet i kravspecifikationen, dette er valgt, fordi det gør robotten mere behagelig at navigere fra GUI'et.\\
Det har ikke været muligt at teste videofeedets framerate (FPS), da der anvendes en Qt-funktion til at vise feedet på GUI'et. 
Videofeedets FPS er dog tilfredsstille til styring af robotten.\\
Det er også svært at teste forbindelsens forsinkelse, så grundet tidsmangel er det valgt ikke at udføre denne test. 
Forsinkelsen er dog tilfredsstillende til styring af robotten.\\
Det blev valgt ikke at udføre en fyldestgørende test på modulporten, da denne ikke er af vital betydning for prototypen. \\
Navigering af GUI'et med øjnene fungere kun under optimale forhold, men beviser proof of concept.


%\end{document}