\chapter{Use Cases}
\label{appendix:usecases}
\section{Use Case 1 - Start system}
\begin{center}
	\begin{tabu} to 1 \textwidth { X[l,1.1]  X[l,3] }
		\tabulinestyle{1pt}
		\tabucline[1pt]{}
		Navn  & Start system  \\
		\tabucline[1pt on2pt]{}
		Mål  & At starte systemet op  \\
		\tabucline[1pt on2pt]{}
		Initiering  & Bruger åbner PC applikation  \\
		\tabucline[1pt on2pt]{}
		Prækondition  & Systemet er i dvaletilstand  \\
		\tabucline[1pt on2pt]{}
		Postkondition  & Systemet er i operativ tilstand  \\
		\tabucline[1pt on2pt]{}
		Hovedscenarie  & \begin{enumerate}
							
							\item PC viser grafisk brugerflade i menuskærmen
							\item PC viser brugermeddelse om, at forbindelsen til robotten er oprettet.				\\			
							$ [ $Undtagelse 1: Forbindelse kunne ikke oprettes$ ] $					
						\end{enumerate}  \\
		\tabucline[1pt on2pt]{}
		Undtagelser & 
			$ [ $Undtagelse 1: Forbindelse kunne ikke oprettes$ ] $
		\begin{enumerate}
			\item PC viser brugermeddelelse om, at forbindelsen ikke kunne oprettes
			\item Use Casen genoptages fra punkt 2
		\end{enumerate}
			  \\
		\tabucline[1pt]{}	
	\end{tabu}
\end{center}


\section{Use Case 2 - Foretag aktivt valg}
\begin{center}
	\begin{tabu} to 1 \textwidth { X[l,1.1]  X[l,3] }
		\tabulinestyle{1pt}
		\tabucline[1pt]{}
		Navn  & Foretag aktivt valg  \\
		\tabucline[1pt on2pt]{}
		Mål  & At have foretaget et aktivt valg  \\
		\tabucline[1pt on2pt]{}
		Initiering  & Bruger kigger på et valgfelt  \\
		\tabucline[1pt on2pt]{}
		Prækondition  & Systemet er i operativ tilstand  \\
		\tabucline[1pt on2pt]{}
		Postkondition  & Bruger har gennemført et aktivt valg  \\
		\tabucline[1pt on2pt]{}
		Hovedscenarie  & \begin{enumerate}			
			\item PC viser rød procesindikator i valgfeltet\\
			$ [ $Undtagelse 1: Bruger kigger væk, før procesindikator er udfyldt$ ] $				
			\item PC viser grønt valgfelt, når procesindikator er udfyldt
		\end{enumerate}  \\
		\tabucline[1pt on2pt]{}
		Undtagelser & $ [ $Undtagelse 1: Bruger kigger væk før procesindikator har nået max$ ] $
		\begin{enumerate}
			\item PC skjuler og nulstiller procesindikator
			\item Use casen afsluttes
		\end{enumerate}
		\\
		\tabucline[1pt]{}	
	\end{tabu}
\end{center}
\newpage

\section{Use Case 3 - Styr robot}
Denne del af funktionaliteten beskriver ved brug af storytelling, hvordan robotten navigeres. 
Brugeren kan navigere robotten, når systemet er i operativ tilstand.
Robotten styres via input som brugeren giver PC'en ved at kigge på valgfelter på skærmen. 
Når brugeren kigger på et kontrolvalgfelt lyser det grønt.\\
\\
\textbf{Frigear}: Ved at se på midten af skærmen, sættes robotten i frigear.\\
\textbf{Fremad}: Ved at se på øverste pil, kører robotten fremad.\\
\textbf{Venstre}: Ved at se på venstre pil, drejer robotten venstre om egen akse.\\
\textbf{Højre}: Ved at se på højre pil, drejer robotten højre om egen akse.\\
\textbf{Brems}: Ved at se på det nederste ikon, bremser robotten.\\
\textbf{Menu}: Når robotten holder stille skiftes bremseikonet ud med et menu-ikon. Her har brugeren mulighed for at tilgå menuen ved "aktivt valg".\\
\\
Menuen indeholder en række punkter, som brugeren vælger ved 'aktivt valg'. Brugeren kan forlade menuen ved at vælge 'tilbage' med 'aktivt valg'.
\newpage

\section{Use Case 4 - Tilgå dvale}
\begin{center}
	\begin{tabu} to 1 \textwidth { X[l,1.1]  X[l,3] }
		\tabulinestyle{1pt}
		\tabucline[1pt]{}
		Navn  & Tilgå dvale  \\
		\tabucline[1pt on2pt]{}
		Mål  & Systemet er i dvaletilstand \\
		\tabucline[1pt on2pt]{}
		Initiering  & Startes af bruger  \\
		\tabucline[1pt on2pt]{}
		Prækondition  & Menuen er åben  \\
		\tabucline[1pt on2pt]{}
		Postkondition  & Applikationen er lukket applikation, og robotten er i dvale  \\
		\tabucline[1pt on2pt]{}
		Hovedscenarie  & \begin{enumerate}
			\item Bruger vælger menupunktet dvale/sluk med 'aktivt valg'
			\item Robotten går i dvale		
			\item PC lukker Applikationen
		\end{enumerate}  \\
		\tabucline[1pt]{}	
	\end{tabu}
\end{center}